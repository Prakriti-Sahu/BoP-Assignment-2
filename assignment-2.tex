\documentclass[journal,12pt,twocolumn]{IEEEtran}
%

\usepackage{setspace}
\usepackage{gensymb}
\singlespacing

\usepackage{amsmath}
\usepackage{amsthm}
\usepackage{txfonts}
\usepackage{cite}
\usepackage{enumitem}
\usepackage{mathtools}
\usepackage{float}
\usepackage{listings}
    \usepackage{color}                                            %%
    \usepackage{array}                                            %%
    \usepackage{longtable}                                        %%
    \usepackage{calc}                                             %%
    \usepackage{multirow}                                         %%
    \usepackage{hhline}                                           %%
    \usepackage{ifthen}                                           %%
  %optionally (for landscape tables embedded in another document): %%
    \usepackage{lscape}     
\usepackage{multicol}
\usepackage{chngcntr}

\renewcommand\thesection{\arabic{section}}
\renewcommand\thesubsection{\thesection.\arabic{subsection}}
\renewcommand\thesubsubsection{\thesubsection.\arabic{subsubsection}}

\renewcommand\thesectiondis{\arabic{section}}
\renewcommand\thesubsectiondis{\thesectiondis.\arabic{subsection}}
\renewcommand\thesubsubsectiondis{\thesubsectiondis.\arabic{subsubsection}}

% correct bad hyphenation here
\hyphenation{op-tical net-works semi-conduc-tor}
\def\inputGnumericTable{}                                 %%

\lstset{
%language=C,
frame=single, 
breaklines=true,
columns=fullflexible
}

\begin{document}
%


\newtheorem{theorem}{Theorem}[section]
\newtheorem{problem}{Problem}
\newtheorem{proposition}{Proposition}[section]
\newtheorem{lemma}{Lemma}[section]
\newtheorem{corollary}[theorem]{Corollary}
\newtheorem{example}{Example}[section]
\newtheorem{definition}[problem]{Definition}
\newcommand{\BEQA}{\begin{eqnarray}}
\newcommand{\EEQA}{\end{eqnarray}}
\newcommand{\define}{\stackrel{\triangle}{=}}
\bibliographystyle{IEEEtran}
\providecommand{\mbf}{\mathbf}
\providecommand{\pr}[1]{\ensuremath{\Pr\left(#1\right)}}
\providecommand{\qfunc}[1]{\ensuremath{Q\left(#1\right)}}
\providecommand{\sbrak}[1]{\ensuremath{{}\left[#1\right]}}
\providecommand{\lsbrak}[1]{\ensuremath{{}\left[#1\right.}}
\providecommand{\rsbrak}[1]{\ensuremath{{}\left.#1\right]}}
\providecommand{\brak}[1]{\ensuremath{\left(#1\right)}}
\providecommand{\lbrak}[1]{\ensuremath{\left(#1\right.}}
\providecommand{\rbrak}[1]{\ensuremath{\left.#1\right)}}
\providecommand{\cbrak}[1]{\ensuremath{\left\{#1\right\}}}
\providecommand{\lcbrak}[1]{\ensuremath{\left\{#1\right.}}
\providecommand{\rcbrak}[1]{\ensuremath{\left.#1\right\}}}
\theoremstyle{remark}
\newtheorem{rem}{Remark}
\newcommand{\sgn}{\mathop{\mathrm{sign}}}
\providecommand{\abs}[1]{\left\vert#1\right\vert}
\providecommand{\res}[1]{\Res\displaylimits_{#1}} 
\providecommand{\norm}[1]{\left\lVert#1\right\rVert}
\providecommand{\mtx}[1]{\mathbf{#1}}
\providecommand{\mean}[1]{E\left[ #1 \right]}
\providecommand{\fourier}{\overset{\mathcal{F}}{ \rightleftharpoons}}
\providecommand{\system}{\overset{\mathcal{H}}{ \longleftrightarrow}}
\newcommand{\solution}{\noindent \textbf{Solution: }}
\newcommand{\cosec}{\,\text{cosec}\,}
\providecommand{\dec}[2]{\ensuremath{\overset{#1}{\underset{#2}{\gtrless}}}}
\newcommand{\myvec}[1]{\ensuremath{\begin{pmatrix}#1\end{pmatrix}}}
\newcommand{\cmyvec}[1]{\ensuremath{\begin{pmatrix*}[c]#1\end{pmatrix*}}}
\newcommand{\mydet}[1]{\ensuremath{\begin{vmatrix}#1\end{vmatrix}}}
\newcommand{\proj}[2]{\textbf{proj}_{\vec{#1}}\vec{#2}}
\let\StandardTheFigure\thefigure
\let\vec\mathbf
\title{
SM5083 - BASICS OF PROGRAMMING
}
\author{ Prakriti Sahu - SM21MTECH12009}
\maketitle
\newpage
\bigskip
\renewcommand{\thefigure}{\theenumi}
\renewcommand{\thetable}{\theenumi}
\renewcommand{\theequation}{\theenumi}
\begin{enumerate}[label=\thesection.\arabic*.,ref=\thesection.\theenumi]
\numberwithin{equation}{enumi}
\section{PROBLEM}
\item Show that the area of the triangle formed by the straight lines: \\
$y=x\tan a$, \\$y=x\tan b$, \\$y=x\tan c + d$ \\is $$\Delta=\frac{d^2 \sin \left ( a-b \right )\cos^{2}c }{2 \sin\left ( b-c \right )\sin \left ( a-c \right ) }$$\\
\solution
The given lines are-
\begin{align}\label{eq:l1}
\begin{split}
y=x\tan a \\ x\tan a-y=0  
\end{split}
\end{align}
\begin{align}\label{eq:l2}
\begin{split}
y=x\tan b \\ x\tan b-y=0  
\end{split}
\end{align}
\begin{align}\label{eq:l3}
\begin{split}
y=x\tan c + d \\ x\tan c-y + d=0  
\end{split} 
\end{align}
For lines represented by the equations:
\begin{align}
a_{1}x+b_{1}y+c_{1}=0
\end{align}
\begin{align}
a_{2}x+b_{2}y+c_{2}=0
\end{align}
\begin{align}
a_{3}x+b_{3}y+c_{3}=0
\end{align}
The area of the triangle enclosed by these lines is given by:
\begin{align}\label{eq:area}
\Delta=\frac{\mydet{a_{1}& b_{1} & c_{1}\\ a_{2} & b_{2} & c_{2}\\ a_{3} & b_{3} & c_{3}}^2}{ 2C_{1}C_{2}C_{3}}
\end{align}\\
where $C_{1},C_{2},C_{3}$ are the co-factors of $c_{1},c_{2},c_{3}$ respectively.\\\\
\item \textbf{Calculating determinant:}\\
Substituting the values from \eqref{eq:l1}, \eqref{eq:l2}, \eqref{eq:l3} to find determinant,
\begin{align}\label{eq:det}
\mydet{\tan a& -1 & 0\\ \tan b& -1 & 0\\ \tan c& -1 & d} \xleftrightarrow[]{R_{1}\leftrightarrow R_{1}-R_{2}}\mydet{\tan a-\tan b& 0 & 0\\ \tan b& -1 & 0\\ \tan c& -1 & d}
\end{align}
Expanding along the first row,
\begin{equation}\label{eq:detex}
determinant=\brak {\tan a-\tan b}\brak{-d}
\end{equation}
\item \textbf{Calculating Co-factors:}\\
From \eqref{eq:det},
\begin{align}\label{eq:c1}
\begin{split}
C_{1}= (-1)^{(1+3)}\mydet{\tan b& -1\\\tan c& -1}&\\=(-1)^4[-\tan b + \tan c] &\\=\tan c - \tan b
\end{split}
\end{align}
\begin{align}\label{eq:c2}
\begin{split}
C_{2}= (-1)^{(2+3)}\mydet{\tan a& -1\\\tan c& -1}&\\=(-1)^5[-\tan a + \tan c] &\\=\tan a - \tan c
\end{split}
\end{align}
\begin{align}\label{eq:c3}
\begin{split}
C_{3}= (-1)^{(3+3)}\mydet{\tan a& -1\\\tan b& -1&\\}&\\=(-1)^6[-\tan a + \tan b] &\\=\tan b - \tan a
\end{split}
\end{align}
Substituting the values from \eqref{eq:detex}, \eqref{eq:c1}, \eqref{eq:c2}, \eqref{eq:c3} into \eqref{eq:area}, we get
\begin{align}
\begin{split}
&\Delta=
\\&\frac{d^2{(\tan a-\tan b)}^2}{2(\tan c-\tan b)(\tan a-\tan c)(\tan b-\tan a)}
\end{split}
\end{align}
Using,
\begin{align}
\tan x=\frac{\sin x}{\cos x}
\end{align}
\begin{align}
\implies\Delta=\frac{-d^2({\frac{\sin a}{\cos a}-\frac{\sin b}{\cos b}})}{2({\frac{\sin c}{\cos c}-\frac{\sin b}{\cos b}})({\frac{\sin a}{\cos a}-\frac{\sin c}{\cos c}})}
\end{align}
\begin{align}
\Delta=\frac{d^2({\frac{\sin a}{\cos a}-\frac{\sin b}{\cos b}})}{2({\frac{\sin b}{\cos b}-\frac{\sin c}{\cos c}})({\frac{\sin a}{\cos a}-\frac{\sin c}{\cos c}})}
\end{align}
\begin{align}
\begin{split}
&\Delta=\\&\frac{d^2({\frac{\sin a \cos b-\sin b \cos a}{\cos a\cos b}})}{2({\frac{\sin b \cos c-\sin c \cos b}{\cos b\cos c}})({\frac{\sin a \cos c-\sin c \cos a}{\cos a\cos c}})}
\end{split}
\end{align}
\begin{align}
\begin{split}
&\Delta\\&=\frac{d^2(\sin a \cos b-\sin b \cos a)(\cos^{2} c)}{2(\sin b \cos c-\sin c \cos b)(\sin a \cos c-\sin c \cos a)}
\end{split}
\end{align}
Using identity-
\begin{align}
\sin{(x-y)}=\sin x \cos y-\sin y \cos x
\end{align}
\begin{align}
\implies\Delta=\frac{d^2\sin (a-b)(\cos^{2} c)}{2\sin (b-c)\sin(a-c)}
\end{align}\\
Hence, proved.
\end{enumerate}
\end{document}
